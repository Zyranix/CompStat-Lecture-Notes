%! TEX root = ../../master.tex
\lecture[I JUST CANT GET ENOUGH OF MCDIARMID]{Mi 15 May 2024}{Typical bounded differences}
We are interested in the number of triangles $f(G)$ of a random graph $G \sim \mathcal{G}_{n,p}$.
Its Lipschitz condition number is given as $n-2$ as one can check easily.

In the following, we want to generalize McDiarmid's theorem to use knowledge of our probabilistic notions.
First, suppose some random variable $X$ with values in $\realnum^n$.
We can relax the Lipschitz condition to only hold for $x,y \in A \subseteq \realnum^n$
such that $A$ satisfies $P(X \in A) = 1$ (but still differing in at most 1 coordinate).
Then, $g(x) \coloneqq f(x) \mathbb{1}_{x \in A} + c_0$ (for suitable $c_0$) still satisfies the original McDiarmid conditions
and we can conclude
\begin{align}
    P(f(X) - \expect{f(X)} \geq \lambda) = P(g(X) - \expect{g(X)} \geq \lambda) \leq \exp\left(\frac{-\lambda^2}{nc^2}\right)
\end{align}
by utilizing our distributional knowledge of $X$.

We can generalize even more and choose $A \subseteq \realnum^n$ arbitrarily instead.
This enables us to split $X$ into two cases depending on if it is included in $A$, and we can write
\begin{equation} \label{eq:upper_bound_split_diarmid}
    \begin{split}
        P & (f(X) - \expect{f(X)} \geq \lambda)                                           \\
        & = P(f(X) - \expect{f(X)} \geq \lambda \mid X \in A ) P(X \in A)               \\
        & \quad + P(f(X) - \expect{f(X)} \geq \lambda  \mid X \not\in A) P(X \not\in A) \\
        & \leq \exp\left(\frac{-\lambda^2}{nc^2}\right) + P(X \not \in A).
    \end{split}
\end{equation}

Now, let us return to counting triangles.
Define for vertices $u,v \in V$ the random variables $X_{u,v}$ denoting the number of
common neighbors of $u$ and $v$, which is binomially distributed according to $Bin(n-2,p^2)$.
The Chernoff nound for binomial random variables tells us for $\delta > 0 $ that
\begin{align*}
    P(X_{u,v} \geq (1+ \delta)\expect{X_{u,v}}) \leq \exp\left(\frac{-\expect{X_{u,v}}\delta^2}{3}\right).
\end{align*}
This implies in our context of triangle counting that
\begin{align*}
    P(X_{u,v} - \expect{X_{u,v}} \geq \delta (n-2)p^2) \leq \exp\left(\frac{-(n-2)p^2\delta^2}{3}\right),
\end{align*}
which, using $\delta = \frac{\lambda}{(n-2)p^2}$ reduces to the more familiar form
\begin{align}\label{eq:upper_bound_mcd_gen}
    P(X_{u,v} - \expect{X_{u,v}} \geq \lambda) \leq \exp\left(\frac{-\lambda^2}{3(n-2)p^2}\right).
\end{align}
We now want to find a non-trivial set $A$ in the sense that \eqref{eq:upper_bound_split_diarmid}
gives us a meaningful upper-bound, i.e. the righthand-side should be smaller than 1.
Define $A$ as the set of graphs such that all pairs of vertices $u,v$ satisfy $X_{u,v} < \expect{X_{u,v}} + \lambda$.
Consider the case $G \not \in A$. Using \eqref{eq:upper_bound_mcd_gen} and union bounding, we see
\begin{align*}
    P(G \not \in A) & = P(\exists u,v: X_{u,v} \geq (n-2)p^2+\lambda)                                                                          \\
                    & = P\left(\bigcup_{u \neq v} X_{u,v} \geq (n-2)p^2 + \lambda \right) \leq n^2 \exp \left(\frac{-\lambda^2}{3np^2}\right).
\end{align*}
If we choose $\lambda = pn^{2+\eps}$ for some small $\eps > 0$, the righthand-side
converges to a value smaller 1.
Notice our choice of $\lambda$ implies $\lambda + np^2 \leq \lambda^2 + \lambda$ as a suitable Lipschitz constant.
Adding the case $P(G \in A)$, we conclude according to \eqref{eq:upper_bound_split_diarmid} for $\mu > 0$
\begin{align*}
    P(f(G) - \expect{f(G)} \geq \mu) \leq \exp\left(\frac{-\mu^2}{n(\lambda+\lambda^2)^2}\right) +  n^2\exp\left(\frac{-\lambda^2}{3np^2}\right).
\end{align*}
Using some confusing blackboard magic and oral remarks we concluded for $\eps > 0$
\begin{align}
    P(f(G) - \expect{f(G)} \geq n^{\frac{1}{2}+\eps}) \xrightarrow{n \rightarrow \infty} 0,
\end{align}
which is a rather strong concentration result for the number of triangles.

Let us summarize the technique we used here as a more general theorem.
\begin{theorem}
    Let $X_1, \dots, X_n$ be i.i.d.~random variables and let $\Lambda \subseteq \realnum$ be
    such that
    \begin{enumerate}
        \item $P(X_i \not\in \Lambda) \leq \exp(- \omega(n^\eps))$ for some $\eps > 0$, and
        \item $|f(x) - f(y)| \leq c$ for all $x,y \in \Lambda^n$ that differ in at most one coordinate.
    \end{enumerate}
    Then
    \begin{align*}
        P(f(X) - \expect{(f(X))} \geq \omega(\sqrt{n})) \xrightarrow{n \rightarrow \infty} 0.
    \end{align*}
\end{theorem}
Finally, we introduce one more important concentration result.
As previously shown, \autoref{thm:distance_concentration} (together with its accompanying remark) yielded some concentration bounds.
Building on this, we can generalize to show following result:
\begin{theorem}[\vocab{Talagrand's inequality}]
    Let $x=(X_1, \dots, X_n)$ be a vector of i.i.d.~random variable in $\realnum$ and $A \subseteq \realnum^n$.
    We define \vocab{Talagrand's Convex Distance} as the distance function
    \begin{align*}
        d_T(x,A) \coloneqq \sup_{||\alpha||=1}d_\alpha(x,A).
    \end{align*}
    Then, it holds that
    \begin{align*}
        d_T(x,A) \geq \frac{1}{\sqrt{n}}d_H(x,A).
    \end{align*}
\end{theorem}
This effectively means any upper bound on $d_T$ is also an upper bound for Hamming distances.