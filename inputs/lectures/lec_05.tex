%! TEX root = ../../master.tex
\lecture[Azuma's inequality. McDiarmid's inequality. Applications.]{Mo 29 Apr 2024}{Martingales and Concentration}
\section{Martingales and Concentration}
In this section we study "bounds" of random variables.
\begin{definition}[\vocab{Bounds}]
    We introduce several notions.
    \begin{itemize}
        \item A \vocab{Tail Bound} is an upper bound for $P(X \geq a)$ given that $X$ is non-negative.
        \item A \vocab{Deviation Inequality} is an upper bound on $P(|X-a| \geq \eps)$
              where $\eps$ usually denotes $\expect{X}$.
    \end{itemize}
\end{definition}
Let us introduce an important inequality.
\begin{theorem}[\vocab{Azuma's Inequality}]
    Let $X$ be a martingale with $X_0 = 0$.
    Suppose that $|X_{n+1} - X_n| \leq 1$ for all $n$.
    Then, for any $\lambda > 0$, it holds that
    \begin{align}
        P(X_n \geq \lambda) \leq \exp{\frac{-\lambda^2}{2n}},
    \end{align}
    and equivalently
    \begin{align}
        P(X_n \geq \lambda \sqrt{n}) \leq \exp{\frac{-\lambda^2}{2}},
    \end{align}
\end{theorem}
\begin{proof}[Proof sketch]
    By properties of a Martingale,
    \begin{align}
        P(X_n \geq \lambda) \leq P(\exp{X_nt} \geq \exp{t\lambda}) \leq \expect{\exp{tX_n}}\exp{-t\lambda}.
    \end{align}
    Consider the telescopic augmentation $X_n = \sum_{i=1}^n (X_i - X_{i-1})$
    and let us apply it to the previous expected value,
    \begin{align}
        \expect{\exp{tX_n}} & = \expect{\underbrace{\exp{t \sum_{i=1}^n (X_i - X_{i-1})}}_{\coloneqq F_n}} = \expect{\prod_{i=1}^{n}\exp{t(X_i - X_{i-1})}} \\
                            & = \expect{\prod_{i=1}^{n-1}F_{i-1}}\expect{\exp{t(X_i - X_{i-1})} \mid F_{i-1} }
    \end{align}
    Notice that $0 \leq \expect{\exp{t(X_i - X_{i-1})} \mid F_{i-1} } \leq \exp{\frac{t^2}{2}}$,
    so using \autoref{thm:hoeffding} on this conditional expected value yields
    \begin{align}
        \expect{\exp{t(X_i - X_{i-1})} \mid F_{i-1} } \leq \exp\left(t\expect{X_n- X_{n-1} \mid F_{n-1} + \frac{t^2}{2}}\right) = \exp\left(\frac{t^2}{2}\right).
    \end{align}
    Notice that we used the Martingale properties in the last step.
\end{proof}
\begin{lemma}[\vocab{Hoeffding's Lemma}] \label{thm:hoeffding}
    If $Y$ is a random variable bounded with $a \leq Y \leq b$, then
    \begin{align}
        \expect{\exp{tY}} \leq \exp{t\expect{Y} + \frac{t^2(b-a)^2}{8}}.
    \end{align}
\end{lemma}
We omit the proof for brevity. Roughly, the idea is to work with hyperbolic functions.

Notice that this version of Azuma's inequality uses quite restrictive requirements,
which we in fact do not need.
Instead, let us formulate following generalization which now allows for more dynamic step sizes and arbitrary start values.
\begin{theorem}
    If $X$ is a martingale and $|X_{n+1} - X_n| \leq a_n$ for all $n$, then
    \begin{align}
        P(|X_n - X_0| \geq \lambda) \leq 2 \exp\left(\frac{-\lambda^2}{2 \sum_{i=1}^{n}a_i^2}\right)
    \end{align}
\end{theorem}
We omit the proof, but it is quite analoguos.

Let us introduce following notion for continuity.
\begin{definition}[Lipschitz condition]
    Given a function $f : \realnum^n \rightarrow \realnum$.
    If there exists a $C \in \realnum$ such that for any $x,y \in \realnum^n$ whose values only differ in one place holds that
    \begin{align}
        |f(x) - f(y)| \leq C
    \end{align}
    we say $f$ satisfies the \vocab{Lipschitz Condition}.
\end{definition}
The Lipschitz condition intuitively tries to limit the derivative in every point of $f$.
In fact, one can show that this is an even stronger condition than "regular" continuity.
\begin{theorem}[\vocab{McDiarmid's Inequality}] \label{thm:mcdiarmid}
    Let $X$ be a stochastic process, and $f : \realnum^n \rightarrow \realnum$ a function satisfying the Lipschitz condition with Lipschitz constant $c$.
    Then, for the \vocab{Doob Martingale}\footnote{The fact that $Z$ is indeed a martingale is a classical result.} defined as
    \begin{align}
        Z_k = \expect{f(X_1, \dots, X_n) \mid \bigotimes_{i=1}^k X_i}
    \end{align}
    we have
    \begin{align} \label{eq:mcdiarmid}
        P(|Z_k - Z_0| \geq \lambda) \leq 2\exp\left(\frac{-2\lambda^2}{kc^2}\right)
    \end{align}
\end{theorem}
Again, we omit the proof, since it does not contain any exciting ideas.
\begin{remark}
    Most of the time, only the special case $k=n$ is interesting.
    (Conditional expectancies are hard to calculate!)
    In this case, the statement \eqref{eq:mcdiarmid} simplifies to
    \begin{align}
        P(f(X_1, \dots, X_n) - \expect{f(X_1, \dots, X_n)} \geq \lambda) \leq \exp\left(\frac{-2\lambda^2}{nc^2}\right).
    \end{align}
    Notice that the absence of the absolute value gets rid of the factor 2 (if needed).
\end{remark}

We now turn to some applications of these results.
In the context of bioinformatics, DNA sequences are the most important research topic.
It might be useful to treat such sequences as random strings, and find interesting patterns in it.
\begin{problem}[Pattern matching]
Find \emph{interesting} substrings of some given random string following an arbitrary unknown distribution.
We say a substring is \emph{interesting} if it occurs more often than it would if the random string followed a uniform distribution.
\end{problem}
Let $N$ be the number of occurences of a fixed substring $B=(B_1, \dots, B_k)$ in the random string $X = (X_1, \dots, X_n)$ given the underlying alphabet $\eta$ with $s \coloneqq |\eta|$.
If $X$ is drawn uniformly, we can easily calculate the expected number of occurences of $B$ as
\begin{align}
    \expect{N} & = \expect{\sum_{i=1}^{n-k+1} \mathbb{1}(X_i = B_i, \dots, X_{i+k} = B_k)} \\
               & = P\left(X_i = B_i, \dots, X_{i+k} = B_k\right) = (n-k+1)\frac{1}{s^k}.
\end{align}
Let us now treat $N : \realnum^n \rightarrow \realnum$ as a function.
Then $|N(x) - N(y)| \leq k$, and define the Doob martingale $Z_k \coloneqq \expect{N \mid \bigotimes_{i=1}^k X_i}$.
This allows us to apply \autoref{thm:mcdiarmid}, and deduce
\begin{align}
    P(|Z_n - Z_0| \geq \lambda) = P(|N - \expect{N}| \geq \lambda) \leq 2\exp\left(\frac{-2\lambda}{nk^2}\right).
\end{align}
