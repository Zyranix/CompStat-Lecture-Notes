%! TEX root = ../../master.tex
\lecture[Path coupling.]{Mo 10 June 2024}{Path coupling.}
\begin{theorem}[Path Coupling]
    Let $Z$ be a Markov chain on a graph $G$ that can only move between adjacent states.
    If $(X,Y)$ is a coupling of $Z$ (for $X,Y$ adjacent) with $\expect{d(X_i,Y_i)}_{X,Y} \leq p$
    where $d$ is the graph distance and $p < 1$, then
    \begin{align*}
        \tau(\eps) \leq \frac{1}{1-p} \ln\left(\frac{\diam(G)}{\eps}\right).
    \end{align*}
\end{theorem}
\begin{proof}
    Use $D \coloneqq \diam(G)$.
    The key fact of the proof is that
    \begin{align}
        (x,y) \mapsto \expect{d(X,Y)}_{X,Y}
    \end{align}
    is a \emph{metric} for $X,Y$ following any distribution on $V$.
    Let $X,Y$ be any vertices on $V$, and let $x = v_0, \dots, v_k, v_{k+1}=y$
    be any path from $x$ to $y$ in $G$.
    Then by assumption
    \begin{align*}
        \expect{d(x_1,y_1)}_{X,Y} \leq \sum_{i=0}^{k} \expect{d(x_i,v_i)}_{v_i,v_{i+1}} \leq \sum_{i=0}^{k} p = p (k+1).
    \end{align*}
    Since the path from $x$ to $y$ was arbitrary, we just take the shortest.
    Then,
    \begin{align}\label{eq:dist_coupling_proof}
        \expect{d(X_1,Y_1)}_{x,y} \leq p d(x,y),
    \end{align}
    and by Markov's inequality
    \begin{align*}
        P_{X,Y}(X_1 \neq Y_1) = P_{X,Y}(d(X_1,Y_1) > 0) \leq \expect{d(X_1,Y_1)}_{X,Y} \leq p d(X,Y).
    \end{align*}
    Thus by Markov property
    \begin{align*}
        P_{x,y}(X_t \neq Y_t) & \leq \expect{d(X_t,Y_t)}_{x,y}
        = \expect{\expect{X_t \neq Y_t \mid X_{t-1},Y_{t-1}}_{x,y}}_{x,y}                                                                                         \\
                              & = \expect{\expect{d(X_1, Y_1)}_{X_{t-1},Y_{t-1}}}_{X,Y}                                                                           \\
                              & \overset{\eqref{eq:dist_coupling_proof}}{\leq} p \cdot \expect{d(X_{t-1}, Y_{t-1})}_{X,Y} \leq \cdots \leq p^t d(x,y) \leq p^t D.
    \end{align*}
    If $p^t D \leq \eps$, then $\tau(\eps) \leq t$, so using $e^x \geq 1+x$
    \begin{align*}
        \exp\left(-(1-p)t\right) \leq (1 - (1-p))^t = p^t \leq \frac{\eps}{D}
    \end{align*}
    and
    \begin{align*}
        t \geq \frac{1}{1-p}\ln\left(\frac{D}{\eps}\right).
    \end{align*}
\end{proof}
\begin{example}
    Let us consider some coupled Markov chains.
    \begin{enumerate}
        \item Let $Z$ be a simple random walk on $K_n$.
              Then $(X,Y)$ always land on the same state except if $X$ chose the position of $Y$.
        \item Let $Z$ be a simple random walk on a complete bipartite graph with self-loops.
              Then (not enough time).
    \end{enumerate}
\end{example}