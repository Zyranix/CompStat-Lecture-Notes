%! TEX root = ../../master.tex
\lecture[More applications of McDiarmid's inequality.]{Mo 06 May 2024}{McDiarmid applications}

\begin{problem}[Balls-into-bins]
Suppose we throw $m$ balls into $n$ bins.
What is the (expected) number of empty bins?
\end{problem}
Let $X_i$ be the bin into which the $i$th ball falls.
Then, for $f : [n]^m \rightarrow [n]$ such that $f$ counts the number of empty bins,
we see $|f(x) - f(y)| \leq 1$, and thus
\begin{align*}
    P(|f(X) - \expect{f(X)}| \geq \lambda) \leq 2 \exp\left(\frac{-2\lambda^2}{m}\right).
\end{align*}
\begin{problem}[Bin packing]
Let $X_1, \dots, X_n$ be random variables with values in $[0,1]$.
What is the (expected) number of bins of capacity 1 needed to pack all $X_i$?
\end{problem}
Define $f : [0,1]^n \rightarrow [n]$ such that $f$ maps into the number of bins needed.
We see $|f(x) - f(y)| \leq 1$, and thus
\begin{align*}
    P(|f(X) - \expect{f(X)}| \geq \lambda) \leq 2 \exp\left(\frac{-2\lambda^2}{m}\right).
\end{align*}
\begin{problem}[Random knapsack]
Let $X_1, \dots, X_n$ be random variables with values in $[0,1]$.
Consider a knapsack of capacity $b$.
A packing of $X$ is a subset of some $X_i$ such that their sum is at most $b$.
We are interested in the \emph{maximal} packing $K:[0,1]^n \rightarrow \realnum$,
i.e. the largest sum of $X_i$'s that is still at most $b$.
\end{problem}
McDiarmid applies as above.

\begin{lemma}
    Let $K_n$ be the complete graph on $n$ vertices.
    Let $(A_1, \dots, A_m)$ be a partition of the edges in $K_n$.
    Furthermore, let $f: \mathcal{G_n} \rightarrow \realnum$ a function defined on the space of all graphs with $n$ vertices,
    and assume $|f(G) - f(H)| \leq C$ if the symmetric difference $E_G \triangle E_H$
    is contained in \emph{one} $A_i$.
    Then, for a random graph $G$ on $n$ vertices
    \begin{align*}
        P(f(G) - \expect{f(G)} \geq \lambda) \leq \exp(\frac{-2 \lambda^2}{mC^2}).
    \end{align*}
\end{lemma}
\begin{proof}
    Trivial by \autoref{thm:mcdiarmid}.
\end{proof}
\begin{corollary}
    If $|f(G) - f(H)| \leq C$ when $G,H$ only differ by one edge, then
    \begin{align*}
        P(f(G) - \expect{f(G)} \geq \lambda) \leq \exp(\frac{-4 \lambda^2}{n(n-1)C^2}).
    \end{align*}
\end{corollary}
\begin{proof}
    Trivial.
\end{proof}
\begin{corollary} \label{thm:cor2_graph_mcdiarmid}
    If $|f(G) - f(H)| \leq C$ when $G,H$ only differ at edges connected to \emph{one} vertex,
    then we can take $m=n$.
\end{corollary}
\begin{proof}
    Use $A_i = \{ \{j,i\}\in E \mid j < i \}$.
\end{proof}
\begin{problem}
Let $G$ be a graph on $n$ vertices.
A coloring of $G$ is a map $c: V_G \rightarrow [n]$
such that neighbors are not colored the same.
The chromatic number $\chi(G)$ of $G$ is the minimal amount of different colors needed
for a valid coloring of $G$.
We are interested in the concentration of the chromatic number.
\end{problem}
\autoref{thm:cor2_graph_mcdiarmid} tells us that
\begin{align*}
    P(|\chi(G) - \expect{\chi(G)}| \geq \lambda) \leq 2\exp\left(\frac{-2\lambda^2}{n}\right).
\end{align*}
\begin{theorem}
    Let $G \sim \mathcal{G}_{n,\frac{1}{2}}$.
    Then, following property for the probability of the concentration for the chromatic number
    holds for every $\eps > 0$:
    \begin{align*}
        P\left(2 \log_2(n)(1 - \eps) \leq \chi(G) \leq \frac{n}{2 \log_2(n)}(1+\eps)\right) \xrightarrow{n \rightarrow \infty} 1.
    \end{align*}
\end{theorem}
Let us outline our core idea first:
Suppose we can find some $s(n)$ such that the probability that ${G}_{n,p}$ has no independent set of size $s(n)$ is small.
Then, we will iteratively color this set, remove it, and apply again $s(n - s(n))$.
In fact, we will upper bound the mentioned probability by $q(n) \coloneqq \exp(-n^{4/3})$.
\begin{proof}
    More detailed, let $\tilde n \coloneqq \frac{n}{\log(n)^2}$.
    We call a set $W \subseteq V_G$ with $|W| \geq \tilde n$ \emph{bad},
    if it contains no independent set of size $s(\tilde n)$.
    We show that the probability that there exists a bad set in $\mathcal{G}_{n,p}$
    is upper-bounded by $2^n q(\tilde n)$, which tends to 0 for $n \rightarrow \infty$.
    The procedure above then lets us iteratively
    \begin{itemize}
        \item color the set of size $s(\tilde n)$,
        \item remove those $s(\tilde n)$ vertices,
    \end{itemize}
    until we are left at most $\tilde n$ vertices and color them uniquely.
    This process uses $\frac{n}{s(\tilde n)}+\tilde n$ colors. \emph{Proof continues next lecture.}
\end{proof}