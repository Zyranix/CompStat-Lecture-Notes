%! TEX root = ../../master.tex
\lecture[Recap of basics.]{Tu 09 Apr 2024}{Introduction}
% \begin{orga}
%      -----
% \end{orga}

\section{Probabilistic method}
This section will introduce how probability can be used to solve problems that at first glance do not have anything to do with probability.
Consider following combinatorial problem.
\begin{definition}[Ramsay numbers]
    We define $R_k$ as the smallest integer $n$ such that any graph with $n$ vertices must contain either a clique or an independent set of size $K$.
    We call $R_k$ the (symmetric) Ramsay numbers.
\end{definition}
Let us first look at some examples to get some feeling.
\begin{example}
    \begin{itemize}
        The first few Ramsay numbers are given as
        \item $R_2=2$: Obviously, $R_k \geq k$. Furthermore, there are only two different graphs with two nodes, i.e. with or without an edge.
              In the former case, both nodes form a clique, in the latter they form an independet set of size 2.
        \item $R_3=6$: First, let us show $R_3 \geq 6$. Consider a cycle of 5 nodes. Then, there is no clique of size 3 (since there is no 3-cycle).
              Also, among any three nodes two nodes are connected by an edge, so there is no independent set. Therefore, this is a counterexample.
              Now, consider a graph with 6 nodes. Suppose there is no clique of size 3. Then it suffices to show that there is an independent set of 3 nodes.
              Indeed, with an ugly case distinction this is possible: If there are no cycles, the graph is bipartite, so there is an independent set of at least 3.
              Otherwise, there exists at least a 4-cycle, but no 3-cycle (i.e. chord-free).
              We can then select at least two independent nodes from the cycle, and if needed the missing third node from the non-cycle nodes such that they form an independent set.
        \item $R_4 = 18$.
        \item $R_5 \in [43, 48]$. The exact value is indeed still unknown!
    \end{itemize}
\end{example}
As we can see, even for small Ramsay numbers, it is not trivial to determine their value.
\begin{theorem}
    For every $k \geq 1$ holds $R_k > 2^{k/2}$.
\end{theorem}
\begin{proof}
    Consider a uniform distribution over all graphs with $n$ vertices (i.e. an Erdös (random) graph $\mathcal{G}(n, \frac{1}{2})$).
    Each edge in particular exists with probability $\frac{1}{2}$.
    Now, have a look at $p \coloneqq P(G \sim \mathcal{G(n, \frac{1}{2})} \text{has a $k$-clique or $k$ independent set})$.
    If we can show that this probability is less than 1, then this means there is an instance such that the property is \emph{not} satisfied,
    and therefore $R_k > n$.

    Let $S$ be a $k$-tuple of vertices. $S$ is per definition a $k$-clique if \emph{all} or \emph{none} of its edges is existent.
    Therefore, its probability of being either is given as
    \begin{align}
        P(S \text{is $k$-clique or $k$-independent set}) = 2 \cdot \frac{1}{2^{\binom{k}{2}}}.
    \end{align}
    The total number of $k$-subsets given $n$ vertices is given as $\binom{n}{k}$, so by basic properties of probability we see
    \begin{align}
        p \leq \binom{n}{k} \cdot \frac{1}{2^{\binom{k-1}{2}}} \leq \frac{n^k}{k!} 2^{1-\frac{k^2-k}{2}}
    \end{align}
\end{proof}

\begin{lemma}\label{thm:existence_rv}
    Let $X$ be a discrete random variable over a set $\Omega$ with $\mathbb{E}[X]= \mu$.
    Then, $P(X \geq \mu) > 0$ and $P(X \leq \mu) >0$.
\end{lemma}
\begin{proof}
    Assume $P(X \geq \mu) = 0$.
    Then $P(X = x) = 0$ for $x \geq \mu$.
    By definition and assumption therefore
    \begin{align}
        \mathbb{E}[X] & = \sum_{x \in X(\Omega)}x P(X=x) = \sum_{x < \mu}x P(X=x)                   \\
                      & < \sum_{x < X(\Omega)}\mu P(X=x) = \mu \sum_{x \in X(\Omega)} P(X=x) = \mu.
    \end{align}
    This is a contradiction!
\end{proof}
Consider now a different problem.
\begin{definition}[Max-Cut]
    Given a graph $G=(V,E)$, find a set $A$ maximizing the number of edges between $A$ and $V \setminus A$.
    We call this the Maximum Cut Problem.
\end{definition}
\begin{theorem}
    Given a graph $G = (V,E)$ with $|E|=m$.
    There exists $A \subset V$ with at least $\frac{m}{2}$ cut size.
\end{theorem}
\begin{proof}
    Choose $A$ uniformly over $\mathcal{P}(V)$, i.e. every node is chosen with probability $\frac{1}{2}$.
    Then, every edge is with probability $\frac{1}{2}$ included in the cut,
    which happens iff exactly one of the nodes of the edge is in $A$.
    Let $X$ be the number of cut edges, and $X_e$ be the indicator variable for $e$ being in the cut.
    By linearity of expectancy,
    \begin{align}
        \mathbb{E}[X]=\mathbb{E}\left[\sum_{e \in E}X_e\right]=\sum_{e \in E}\mathbb{E}[X_e]=m \cdot \frac{1}{2}.
    \end{align}
    Using \autoref{thm:existence_rv} there is positive probability for a choice of $A$ having cut size at least $m/2$.
\end{proof}
How can we use these results to actually construct a solution?
\begin{definition}[Las-Vegas Algorithm]
    Let $T$ be the run-time of an algorithm.
    There are two basically equivalent definitions, i.e. we call a Las-Vegas Algorithm an algorithm which
    \begin{enumerate}
        \item always returns the correct answer
              with $\mathbb{E}[T] \in \mathcal{O}(n^k)$, or
        \item has runtime always in $\mathcal{O}(n^k)$ and returns a correct answer with probability at least $\delta$
              (and otherwise none).
    \end{enumerate}
\end{definition}
\begin{remark}
    The equivalence stems from the fact that the second variant can be seen as a geometric process:
    Consider the number $T$ of attempts until the algorithm returns a correct result.
    Then, $\mathbb{E}[T] = \frac{1}{1-\delta}$, so the total runtime is still $\mathcal{O}(n^k)$ in expectancy.
\end{remark}
Turning back to the Max-Cut problem, let us now develop an algorithm
that constructively finds a set $A$ with a large cut (i.e. at least $\frac{m}{2}$).

Interestingly, we do not even need a Las-Vegas algorithm to find a big cut.
Instead, we can construct using probabilistic arguments a deterministic algorithm running in polynomial time.
The idea is to greedily decide for each vertex if we want it in our cut set or not
based on a case distinction using conditional expectation.
\begin{theorem}
    Given a graph $G = (V,E)$.
    Let $A \sim \mathcal{U}_{\mathcal{P}(V)}$, $C_A$ the amount of cuts,
    and $x_1, \cdots, x_n$ indicate the position of the corresponding vertex.
    Then $\expect{C_A \mid x_1, \cdots, x_k}$ is the expected number of cuts given the positions
    Then \autoref{algo:bigcut_det} satisfies following statements:
    \begin{enumerate}
        \item $\mathbb{E}[X_A \mid x_1, \cdots, x_k] \leq \mathbb{E}[X_A \mid x_1, \cdots, x_k, x_{k+1}]$
        \item The algorithm runs in $\mathcal{O}(n+m)$
              and returns a big cut of size at least $m/2$.
    \end{enumerate}
\end{theorem}
First, have a look at our proposed algorithm:
\begin{algorithm}[H]
    \label{algo:bigcut_det}
    \SetAlgoLined
    $A \leftarrow \emptyset, B \leftarrow \emptyset$
    \For{$k=1,\cdots,n$}{
        \If{$|\{(v_k,u) \in E \mid u \in A\}| \leq |\{(v_k,u) \in E \mid u \in B\}|$}{
            $A \leftarrow A + v_k$
        }\Else {
            $B \leftarrow B + v_k$
        }
    }
    return $A$
    \caption{Find Big-Cut of $G = (V,E)$}
\end{algorithm}
The
\begin{proof}

\end{proof}


If we choose $A$ again uniformly distributed over $\mathcal{P}(A)$,
then $P(X_A \geq \frac{m}{2}) = p$.

Let us try to find a more meaningful way of expressing $p$:
\begin{align}
    \mathbb(E)[P(X_A \geq \frac{m}{2})] & = \frac{m}{2} = \sum_{i < \frac{m}{2}}i \cdot P(X_A = i) + \sum_{i \geq \frac{m}{2}}i \cdot P(X_A = i) \\
                                        & \leq (\frac{m}{2} - 1)(1 - p) + mp
\end{align}
Therefore, $p \geq \frac{1}{\frac{m}{2} + 1}$.

Let $v_1, \cdots, v_n$ be indicator variables for a vertex $k \in A$. Then for $k < n$
\begin{align}
    \mathbb{E}(X_A \mid v_1, \cdots, v_k) = \frac{1}{2}\mathbb{E}[X_A \mid v_1, \cdots, v_k, v_{k+1} = 1] + \frac{1}{2}\mathbb{E}[X_A \mid v_1, \cdots v_k, v_{k+1} = 0]
\end{align}
We can now greedily fix $v_{k+1}$ to be the choice with higher expectancy.
$\leq \mathbb{E}(X_A \mid v_1, \cdots, v_k, v_{k+1})$
\todo{Look up algo and finish section}
