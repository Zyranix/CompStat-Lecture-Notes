%! TEX root = ../../master.tex
\lecture[Probabilistic method using independent events. Lovász local lemma.]{Mi 17 Apr 2024}{Independent events}
% \begin{orga}
%      -----
% \end{orga}

\begin{definition}[$\SAT$ problem]
    Given a boolean formula in $k$-CNF (conjunctive normal form).
    The decision problem if there is an assignment that the formula is true is called $k$-$\SAT$.
\end{definition}
Assume we have a $k$-$\SAT$ instance $\phi$, and assume each variable appears in exactly one clause.
Then it is possible to show $\phi \in \SAT$.
\begin{example}
    For $3$-$\SAT$ consider $(x_1 \OR x_2 \OR \NOT x_3) \AND (x_4 \OR \NOT x_5 \OR x_6)$.
    Then, a valid assignment is $x_1 = x_2 = x_3 = 1, x_4=x_5=x_6=0$.
\end{example}
It is rather intuitive this statement must hold, since we enforce some form of indepence.
This makes it interesting to look at from the lense of the probability, and its notion of indepence.

So, let us assume a uniform random assignment $\alpha$ for such a $\phi$ (with $n$ clauses).
Let $E_i$ be the event that clause $i$ is not satisfied by $\alpha$.
Then, $\bigcup E_i$ is the event that $\phi$ is not satisfied by $\alpha$, and thus
$\bigcap \overline{E_i}$ denotes $\phi$ being satisfied by $\alpha$.
Similarly to previous applications of the probabilistic method we see that
\begin{align}
    P\left(\bigcap_{i=1}^n \overline{E_i}\right) = \prod_{i=1}^n P(\overline{E_i}) = \prod_{i=1}^{n}1 - 2^{-k} > 0
\end{align}
shows that there \emph{are} assignments that satisfy $\alpha$.
However, we were now able to use indepence of $E_i$ in this case.

While this example does not seem too spectacular at first, we can now try to leviate our conditions
and only assume limited independence.
Let us introduce a new construct.
\begin{definition}[Dependency graph]
    Consider a set of events $E_1, \dots, E_n$.
    We call $G = (V,E)$ with $V= \{1, \dots, n\}$ the \vocab{Dependency Graph} if $E$ satisfies that
    every $E_i$ is mutually independent of the set $\{E_j \mid (i,j) \neq E_i \}$.
\end{definition}

\begin{lemma}
    If events $E_1, \dots, E_n$ are mutually independent, then its counterparts $\overline{E_1}, \dots, \overline{E_n}$ are mutually independent.
\end{lemma}
Using these notions, we can now introduce a rather strong tool!
\begin{theorem}[Symmetric \vocab{Lovász Local Lemma}]
    \label{thm:lll}
    Let $E_1, \dots, E_n$ be a set of events and assume
    \begin{enumerate}
        \item $P(E_i) \leq p$ for some fixed $p$,
        \item the maximum degree of the dependency graph of these events is $d$, and
        \item $4dp \leq 1$.
    \end{enumerate}
    Then $P(\bigcap_{i=1}^n \overline{E_i}) > 0$.
\end{theorem}
The proof is rather technical, but the main idea is to use two nested inductions,
and rewrite our result as a fancy product using conditional probabilities.
\begin{proof}
    Let $S \subseteq \{1, \dots, n\}$.
    We will show by induction on the size $s$ of $S$ that for all $k \not \in S$ it holds
    \begin{align}
        P(E_k \mid \bigcup_{j \in S} \overline{E_j}) \leq 2p.
    \end{align}
    For the base case $s=0$ by assumption $P(E_k) \leq p \leq 2p$.

    For the induction step we first need to show $P(\bigcap_{j \in S} \overline{E_j}) > 0$.
    Again, we use another induction.
    For $s=1$ this is clear by assumption. (Notice $p < 1$).
    Now, w.l.o.g. let $S=\{1,\dots, s\}$.
    By definition of conditional expectancy,
    \begin{align*}
        P(\bigcap_{j \in S} \overline{E_j}) & = P(\overline{E_s} \mid \bigcap_{j = 1}^{s-1} \overline{E_j}) \cdot P(\bigcap_{i=1}^{s-1}\overline{E_j})                                                                                                   \\
                                            & = \underbrace{\left(1 - P(E_s \mid \bigcap_{j = 1}^{s-1}\overline{E_j})\right)}_{>0 \text{ by outer induction}}\cdot \underbrace{P(\bigcap_{i=1}^{s-1}\overline{E_j})}_{>0 \text{ by inner induction}} > 0
    \end{align*}
    This concludes the inner induction, so let us continue with the outer induction step.
    Let us split $S$ into $S_1= \{j \in S \mid (k,j) \in E\},S_2 = S \setminus S_1$.
    Denote with $F_k = \bigcup_{i=1}^k \overline{E_k}$.
    Then
    \begin{align*}
        P(E_k \mid F_s) = \frac{P(E_k \cap F_s)}{P(F_s)} = \frac{P(E_k \cap F_{S_1} \cap F_{S_2})}{P(F_{S_1} \cap F_{S_2})} = \frac{P(E_k \cap F_{S_1} \mid F_{S_2})}{P(F_{S_1} \mid F_{S_2})}
    \end{align*}
    Consider both parts of the final fraction, and let us bound them:
    \begin{align*}
        P(E_k \cap F_{S_1} \mid F_{S_2}) & \leq P(E_k \mid F_{S_2}) \leq p                                                                                              \\
        P(F_{S_1} \mid F_{S_2})          & = 1 - P(\overline{F_{S_1}} \mid F_{S_2}) = 1 - P(\bigcup_{i \in S_{1}}E_i \mid F_{S_2})                                      \\
                                         & \geq 1- \sum_{i \in S_1} P(E_i \mid \bigcap_{j \in S_2}\overline{E_j}) \geq 1 - |S_1| \cdot 2p \geq 1 - 2pd \geq \frac{1}{2}
    \end{align*}
    Applying these results yields $2p$ as an upper bound for $P(E_k \mid F_s)$, which also concludes the outer induction.

    Continuing with the actual statement, we get
    \begin{align*}
        P(\bigcap_{i=1}^n \overline{E_i}) = P(\overline{E_n} \mid \bigcap_{i=1}^{n-1} \overline{E_i}) \cdot P(\bigcap_{i=1}^{n-1} \overline{E_i}) =\prod_{i=1}^n P(\overline{E_i} \mid \bigcap_{j=1}^{i-1} \overline{E_i}) \geq (1-2p)^{n} > 0.
    \end{align*}


\end{proof}
Turning back to our $k$-$\SAT$ problem, we are now able to show a stronger version.
\begin{theorem}
    Given a $k$-$\SAT$ instance $\phi$ (with $n$ clauses), and
    assume that no variable appears in more than $\frac{2^k}{4k}$ clauses.
    Then, $\phi \in \SAT$.
\end{theorem}
\begin{proof}
    We motivate \autoref{thm:lll}.
    Firstly, we see $P(E_i) \leq 2^{-k}$.
    For each vertex of the dependency graph its degree is at most $\frac{2^k}{4k}\cdot k$.
    Therefore, $4 \cdot 2^{-k} \cdot \frac{2^k}{4k}k \leq 1$, and $P(\bigcap_{i=1}^n \overline{E_i}) > 0$ implies the existence of a valid assignment.
\end{proof}
Some more applications.
\begin{theorem}
    Assume $n$ pairs of vertices need to be connected using $n$ disjoint paths on
    a given network $E$.
    Each pair $i$ can choose from a collection $F_i$ of $m$ paths.
    If any path in $F_i$ shares edges by at most $k$ paths in $F_j$
    and $\frac{8nk}{m} \leq 1$, then we can always choose an edge-disjoint collection of paths.
\end{theorem}
\begin{proof}
    Consider a probability space where every pair $i$ chooses a path in $F_i$ uniformly distributed, i.e. with probability $1/m$.
    We define $E_{i,j}$ as the "bad" event that paths $i,j$ share any edges,
    which occurs with probability $k/m$.
    The degree of the corresponding dependency graph then is $2n$.
    By \autoref{thm:lll}, we are done.

\end{proof}